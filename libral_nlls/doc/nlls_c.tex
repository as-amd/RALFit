\documentclass{spec}

\usepackage{amsfonts,bm,amsmath}
\usepackage{verbatim}
\usepackage{algorithm, algpseudocode}
\usepackage{caption}

% set the release and package names
\newcommand{\libraryname}{RAL}
\newcommand{\packagename}{NLLS}
\newcommand{\fullpackagename}{\libraryname\_\packagename}
\newcommand{\versionum}{0.5.0}
\newcommand{\versiondate}{15 January 2016}
\newcommand{\version}{\versionum}
\newcommand{\vx}{ {\bm x} } % macro for a vector x
\newcommand{\vr}{ {\bm r} } % macro for a vector r
\newcommand{\vg}{ {\bm g} } % macro for a vector g
\newcommand{\vd}{ {\bm d} } % macro for a vector d
\newcommand{\vp}{ {\bm p} } % macro for a vector p                 
\newcommand{\vw}{ {\bm w} } % macro for a vector w                
\newcommand{\vH}{ {\bm H} } % macro for a matrix H
\newcommand{\vD}{ {\bm D} } % macro for a matrix D
\newcommand{\vhess}{ {\bm H_f} } % macro for a matrix H
\newcommand{\vJ}{ {\bm J} } % macro for a matrix J
\newcommand{\tx}{ {\tt x} } % macro for a vector x
\newcommand{\tr}{ {\tt r} } % macro for a vector r
\newcommand{\tg}{ {\tt g} } % macro for a vector g
\newcommand{\td}{ {\tt d} } % macro for a vector d
\newcommand{\ty}{ {\tt y} } % macro for a vector d
\newcommand{\tH}{ {\tt H} } % macro for a matrix H
\newcommand{\vW}{ {\bm W} } % macro for a matrix W
\newcommand{\thess}{ {\tt Hf} } % macro for a matrix H
\newcommand{\tJ}{ {\tt J} } % macro for a matrix J
\newcommand{\iter}[2][k]{ #2_{#1}^{}} % macro for an iteration
\newcommand{\comp}[2][i]{ #2_{#1}^{}} % macro for a component of a vector

\newcommand{\ct}{.}
\begin{document}

\hslheader

\begin{center}
\huge \sc  C Interface
\end{center}

\hslsummary

% The beginning summary section
{\tt \fullpackagename} computes a solution $\vx$ to the non-linear least-squares problem
\begin{equation}
\min_\vx \  F(\vx) := \frac{1}{2}\| \vr(\vx) \|_{\vW}^2 + \frac{\sigma}{p}\| \vx\|_2^p,
\label{eq:nlls_problem}
\end{equation}
where $\vW\in\mathbb{R}^{m\times m}$ is a diagonal, non-negative, weighting matrix, and $\vr(\vx) =(\comp[1]{r}(\vx), \comp[2]{r}(\vx),...,\comp[m]{r}(\vx))^T$ is a non-linear function.

A typical use may be to fit a function $f(\vx)$ to the data $y_i, \ t_i$, weighted by the uncertainty of the data, $\sigma_i$, so that
$$r_i(\vx) := y_i - f(\vx;t_i),$$
and $\vW$ is the diagonal matrix such that $\vW_{ii} = (1/\sqrt{\sigma_i}).$  For this reason 
we refer to the function $\vr$ as the \emph{residual} function.
% the fit of the data $y$ to some non-linear function ${\bm f} : \mathbb{R}^n \rightarrow \mathbb{R}^m$
% ($m>n$).
% The $n$ variables that are fitted are $\vx=(x_1,x_2,...,x_n)^T$.
% \textcolor{blue}{Some confusion: the $y_i$ don't appear again.}

The algorithm is iterative.
At each point, $\iter{\vx}$, the algorithm builds a model of the function at the next step, $F({\iter{\vx}+\iter{\vs}})$, which we refer to as $m_k(\cdot)$.  We allow either a Gauss-Newton model, a (quasi-)Newton model, or a Newton-tensor model; see Section \ref{sec:model_description} for more details about these models.  

Once the model has been formed we find a candidate for the next step by either solving a trust-region sub-problem of the form
\begin{equation}
\iter{\vs} = \arg \min_{\vs} \ \iter{m} (\vs) \quad \mathrm{s.t.} \quad  \|\vs\|_B \leq \Delta_k,\label{eq:tr_subproblem}
\end{equation}
or by solving the regularized problem 
\begin{equation}
\iter{\vs} = \arg \min_{\vs} \ \iter{m} (\vs)  + \frac{1}{\Delta_k}\cdot \frac{1}{p} \|\vs\|_B^p,\label{eq:reg_subproblem}
\end{equation}
where $\Delta_k$ is a parameter of the algorithm (the trust region radius or the inverse of the regularization parameter respectively), $p$ is a given integer, and $B$ is a symmetric positive definite weighting matrix that is calculated by the algorithm.
The quantity
\[\rho = \frac{F(\iter{\vx}) - F(\iter{\vx} + \iter{\vs})}{\iter{m}(\iter{\vx}) - \iter{m}(\iter{\vx} + \iter{\vs})}\]
is then calculated.
If this is sufficiently large we accept the step, and $\iter[k+1]{\vx}$ is set to $\iter{\vx} + \iter{\vs}$; if not, the parameter $\Delta_k$
is reduced and  the resulting new trust-region sub-problem is solved.  If the step is very successful -- in that $\rho$ is close to one --
$\Delta_k$ is increased.

This process continues until either the residual, $\|\vr(\iter{\vx})\|_\vW$, or a measure of the gradient,
$\|{\iter{\vJ}}^T\vW\vr(\iter{\vx})\|_2 / \|\vr(\iter{\vx})\|_\vW$, is sufficiently small.


%%% Local Variables: 
%%% mode: latex
%%% TeX-master: "nlls_fortran"
%%% End: 


%!!!!!!!!!!!!!!!!!!!!!!!!!!!!
\hslattributes
\hslversions{\versionum\ (\versiondate)}.
\hslIRDCZ Real (single, double).
\hsllanguage Fortran 2003 subset (F95+TR155581).
\hsldate January 2016.
\hslorigin The Numerical Analysis Group, Rutherford Appleton Laboratory.
\hslremark The development of this package was
partially supported by EPSRC grant EP/M025179/1.

%!!!!!!!!!!!!!!!!!!!!!!!!!!!
\newpage
\hslhowto

\subsection{Calling sequences}

Functions signatures are defined in a header file
\begin{verbatim}
   #include "ral_nlls.h"
\end{verbatim}
\medskip

\noindent The user callable subroutines are:
\vspace{-0.1cm}
\begin{description}
   \item[\texttt{ral\_nlls\_default\_options()}] initializes solver options to default values.
   \item[\texttt{nlls\_solve()}]  solves the non-linear least squares problem (Section~\ref{eq:nlls_problem}).
   \item[\texttt{ral\_nlls\_init\_workspace()}] initialises a workspace for use with \texttt{ral\_nlls\_iterate()}.
   \item[\texttt{ral\_nlls\_iterate()}] performs a single iteration of the solver algorithm.
   \item[\texttt{ral\_nlls\_free\_workspace()}] frees memory allocated by a call to \texttt{ral\_nlls\_init\_workspace()}.
\end{description}

%%%%%%%%%%%%%%%%%%%%%% derived types %%%%%%%%%%%%%%%%%%%%%%%%

\hsltypes
\label{derived types}
For each problem, the user must employ the derived types defined by the
module to declare scalars of the types {\tt struct ral\_nlls\_options}, and
{\tt ral\_nlls\_inform}.
The following pseudocode illustrates this.
\begin{verbatim}
   #include "ral_nlls.h"
   ...
   struct ral_nlls_options options;
   struct ral_nlls_inform inform;
   ...
\end{verbatim}
The members of these structs are explained
in Sections~\ref{typeoptions} and \ref{typeinform}.


%%%%%%%%%%%%%%%%%%%%%% argument lists %%%%%%%%%%%%%%%%%%%%%%%%
\hslarguments

\subsubsection{Integer and package types}
%{\tt INTEGER} denotes default {\tt INTEGER} and
%{\tt INTEGER(long)} denotes {\tt INTEGER(kind=selected\_int\_kind(18))}.
The term {\bf package type} is used to mean \texttt{float}
if the single precision version is being used and
\texttt{double} for the double precision version.

\subsubsection{To initialise members of \texttt{struct nlls\_options} to default values}

To initialise the value of \texttt{struct nlls\_options}, the user \textbf{must} make a
call to the following suboutine (failure to do so will result in undefined behaviour):
\begin{verbatim}
   void ral_nlls_default_options(struct ral_nlls_options *options);
\end{verbatim}

\begin{description}
   \itt{*options} will be initialised to default values on return.
\end{description}

\subsubsection{To solve the non-linear least squares problem}
\label{sec:solve}

To solve the non-linear least squares problem make a call of the following
subroutine:

\begin{verbatim}
   void nlls_solve( int n, int m, double X[], ral_nlls_eval_r_type eval_r,
      ral_nlls_eval_j_type eval_j, ral_nlls_eval_hf_type eval_hf,
      void* params, struct nlls_options const* options, struct nlls_inform* inform,
      double weights[])
\end{verbatim}

\begin{description}
\itt{n} holds the number $n$ of
variables to be fitted; i.e., $n$ is the length of the unknown vector $\bm x$.

\itt{m} holds the number $m$ of
data points available; i.e., $m$ is the number of functions $f_i$.
\textbf{Restriction:} \texttt{m},\texttt{n}$>$\texttt{0}

\itt{x} must hold the initial guess for $\bm x$, and on
successful exit it holds the solution to the non-linear least squares problem.

\itt{eval\_r} specifies a callback function that, given a point $\iter{\vx}$,
returns the vector $\vr(\iter{\vx})$. Details of the function signature and
requirements are are given in Section~\ref{sec::function_eval}.

\itt{eval\_j} specifies a callback function that, given a point $\iter{\vx}$,
returns the $m \times n$ Jacobian matrix, $\iter{\vJ}$, of $\vr$ at $\iter{\vx}$. Details of the function signature and requirements are are given in
Section~\ref{sec::function_eval}.

\itt{eval\_hf} is a {\tt PROCEDURE} that, given a point $\iter{\vx}$
and function $\vr(\iter{\vx})$, returns the second-order terms of the Hessian at $\iter{\vx}$.
Further details of the format required are given in Section~\ref{sec::function_eval}.

\itt{params} is a pointer to user data that is passed unaltered to the callback
functions {\tt eval\_r}, {\tt eval\_J}, and {\tt eval\_Hf}.

\itt{inform} provides information about the execution
of the subroutine, as explained in Section~\ref{typeinform}.

\itt{options} specifies options that control the execution of the subroutine,
see Section~\ref{typeoptions}.

\itt{weights} may be {\tt NULL}, otherwise it is a rank-1 array of size {\tt m}. If present, {\tt weights} holds the square-roots of the 
diagonal entries of the weighting matrix, $\vW$, in (\ref{eq:nlls_problem}).  If absent, then the norm in (\ref{eq:nlls_problem}) is taken to be the 2-norm, that is, $\vW = I$.
\end{description}

\subsubsection{To initialise a workspace for use with \texttt{ral\_nlls\_iterate()}}

Prior to the first call of \texttt{ral\_nlls\_iterate()}, the workspace must be
initialised by a call to the following subroutine:
\begin{verbatim}
   void ral_nlls_init_workspace(void **workspace);
\end{verbatim}

\begin{description}
   \itt{*workspace} will, on return, be allocated and initialised using Fortran intrinsics.
      To avoid a memory leak, it must be freed through a call to \texttt{ral\_nlls\_free\_workspace()}.
\end{description}

\subsubsection{To iterate once}
\label{sec:iterate}
Alternatively, the user may step through the solution process one iteration at
a time by making a call of the following form:

\begin{verbatim}
   void ral_nlls_iterate( int n, int m, double X[], void* workspace,
      ral_nlls_eval_r_type eval_r, ral_nlls_eval_j_type eval_j, ral_nlls_eval_hf_type eval_hf,
      void* params, struct nlls_options const* options, struct nlls_inform* inform,
      double weights[])
\end{verbatim}

\begin{description}

\item[\normalfont \texttt{n}, \texttt{m}, \texttt{eval\_F}, \texttt{eval\_J}, \texttt{eval\_HF}, \texttt{params}, \texttt{options}, \texttt{inform}, and \texttt{weights}] are as described in Section~\ref{sec:solve}.

\itt{X} is an array of size {\tt n}. On the first call, it must hold the initial guess for
$\bm x$. On return it holds the value of $\bm x$ at the current iterate, and
must be passed unaltered to any subsequent call to \texttt{ral\_nlls\_iterate()}.

\itt{w} is workspace allocated and initialised through a previous call to
\texttt{ral\_nlls\_init\_workspace()}.

\end{description}

\subsubsection{To free a workspace when it is no longer required}

Memory allocated during the call to \texttt{ral\_nlls\_init\_workspace()} may be freed
by a call to the following subroutine:
\begin{verbatim}
   void ral_nlls_free_workspace(void **workspace);
\end{verbatim}

\begin{description}
   \itt{*workspace} is the workspace to be freed. On exit it will be set to \texttt{NULL}.
\end{description}


\subsection{User-supplied function evaluation routines}
\label{sec::function_eval}
In order to evaluate the function, Jacobian and Hessian at a point, the user
must supply callback functions that perform this operation that the code
{\tt ral\_nlls} will call internally.

In order to pass user-defined data into the evaluation calls, the parameter
\texttt{params} is passed unaltered to the callback functions. Typically this
will be a pointer to a user defined structure that stores the data to be fitted.

\subsubsection{For evaluating the function $r(x)$}
A subroutine must be supplied to calculate $r(x)$ for a given vector $x$. It
must have the following signature:

\begin{verbatim}
   int eval_r (int n, int m, void const* params, double const* x, double* r);
\end{verbatim}

\begin{description}
   \itt{n, m, params} are passed unchanged as provided in the call to
      {\tt ral\_nlls}.

   \itt{x} holds the current point $\iter{\vx}$ at which to evaluate $\vr(\iter{\vx}$.

   \itt{r} must be set by the routine to hold the residual function
      evaluated at the current point $\iter{\vx}$, $\vr(\iter{\vx})$.
\end{description}
\textbf{Return value:} The function should return \texttt{0} on success. A
non-zero return value will cause the least squares fitting to abort with an
error code.


\subsubsection{For evaluating the function $J = \nabla \vr(\iter{\vx})$}
A subroutine must be supplied to calculate $J = \nabla \vr(\iter{\vx})$ for a given vector $x$. It must have the following signature:

\begin{verbatim}
   int eval_j (int n, int m, void const* params, double const* x, double* J);
\end{verbatim}

\begin{description}
   \itt{n, m, params} are passed unchanged as provided in the call to
      {\tt ral\_nlls}.

   \itt{x} holds the current point $\iter{\vx}$ at which to evaluate
      $J(\iter{\vx})$.

   \itt{J} must be set by the routine to hold the Jacobian of the residual
      function evaluated at the current point $\iter{\vx}$, $\vr(\iter{\vx})$.
      \texttt{J[i*m+j]} must be set to hold $\nabla_{x_j} r_i(\iter{\vx})$.
\end{description}
\textbf{Return value:} The function should return \texttt{0} on success. A
non-zero return value will cause the least squares fitting to abort with an
error code.

\subsubsection{For evaluating the function $HF = \sum_{i=1}^m \vr_i(x) \nabla^2 \vr_i(x)$}
A subroutine must be supplied to calculate $HF = \sum_{i=1}^m \vr_i \nabla^2 \vr_i(x)$ for given vectors $x \in \mathbb{R}^n$ and $r \in \mathbb{R}^m$. It must have the following signature

\begin{verbatim}
int eval_hf (int n, int m, void const* params, double const* x, double const* r, double* *HF);
\end{verbatim}

\begin{description}
   \itt{n, m, params} are passed unchanged as provided in the call to
      {\tt ral\_nlls}.

   \itt{x} holds the current point $\iter{\vx}$ at which to evaluate $rx)$.

   \itt{r} holds $\vr(x)$, as returned by a previous call to \texttt{eval\_r}.

   \itt{HF} must be set by the routine to holds the matrix
      $\sum_{i = 1}^m \comp{\vr}(\iter{\vx})\nabla^2\comp{\vr}(\iter{\vx})$, held
      by columns as a vector.
\end{description}
\textbf{Return value:} The function should return \texttt{0} on success. A
non-zero return value will cause the least squares fitting to abort with an
error code.
%%%%%%%%%%%%%%%%%%%%%%%%%%%%%%%%%%%%%%%%%%%%%%%%%%%%%%%%%%%%%%%%%%%%%%



\subsection{The options derived data type}
\label{typeoptions}

The structure of type {\tt struct ral\_nlls\_options} is used to hold
controlling data. The components must be initialised through a call to
\texttt{ral\_nlls\_default\_options()}.

\vspace{2mm}

\noindent {\bf Components that control printing}
\begin{description}

\itt{int error} with default value {\tt 6}
is used as the Fortran unit number for error messages. If it is negative, these
messages will be suppressed.

\itt{int out} with default value {\tt 6}
is used as the Fortran unit number for general messages. If it is negative, these messages will be suppressed.

\itt{int print\_level} with default value {\tt 0} that
controls the level of output required.
\begin{description}
\item{\tt $\leq$ 0} No informational output will occur.
\item{\tt = 1} As 0, plus gives a one-line summary for each iteration.
\item{\tt = 2} As 1, plus gives a summary of the inner iteration for each iteration.
\item{\tt > 3} As 3, and gives increasingly verbose (debugging) output.
\end{description}
The default is {\tt diagnostics\_level} $=$ 0.
\end{description}

\noindent {\bf Components that control the main iteration}.

\begin{description}

\itt{int maxit} gives an upper bound on the number
of iterations the algorithm is allowed to take before being stopped.  The default value is {\tt 100}.

\itt{int model} specifies the model, $m_k(\cdot)$, used.  Possible values are
\begin{description}
  \item{\tt 1} Gauss-Newton (no Hessian).
  \item{\tt 2} Newton (exact Hessian).
%  \item{\tt 3} Barely second-order (Hessian matrix approximated by the identity).
  \item{\tt 3} Hybrid method (mixture of Gauss-Newton/Newton as appropriate).
\end{description}
The default is {\tt model = 1}.

\itt{int nlls\_method} specifies the method used to solve
(or approximate the solution to) the trust-region sub problem.  Possible values are
\begin{description}
  \item{\tt 1} Powell's dogleg method (approximates the solution).
  \item{\tt 2} The Adachi-Iwata-Nakatsukasa-Takeda (AINT) method.
  \item{\tt 3} The More-Sorensen method.
  \item{\tt 4} Galahad's DTRS method
\end{description}
The default is {\tt nlls\_method = 1}.

\itt{double stop\_g\_absolute} specifies the absolute tolerance for convergence.

\itt{double stop\_g\_relative} specifies the relative tolerance for convergence.

\itt{double relative\_tr\_radius} specifies whether the initial trust region
radius should be scaled.

\itt{double initial\_radius\_scale} specifies the scaling parameter for the initial trust region radius, which is only used if {\tt relative\_tr\_radius = 1}.

\itt{double initial\_radius} specifies the initial trust-region radius, $\Delta$.

\itt{double maximum\_radius} specifies the maximum size permitted for the trust-region radius.

\itt{double eta\_successful} specifies the smallest value of $\rho$ such that we accept the step.

\itt{double eta\_very\_successful} specifies the value of $\rho$ after which we increase the trust-region radius.

\itt{double eta\_too\_successful} specifies that value of $\rho$ after which we accept the step,
but keep the trust-region radius unchanged.

\itt{double radius\_increase} specifies the factor to increase the trust-region radius by.

\itt{double radius\_reduce} specifies the factor to decrease the trust-region radius by.

\itt{double hybrid\_switch} specifies the value, if {\tt nlls\_method = 9},
at which we switch to second derivatives.

\itt{bool exact\_second\_derivatives} if {\tt true}, signifies that the
exact second derivatives are available (and, if {\tt false}, approximates them using a secant method).

\itt{int more\_sorensen\_maxits} if {\tt nlls\_method = 3}, specifies the maximum number of iterations allowed in the More-Sorensen method.

\itt{double more\_sorensen\_shift} if {\tt nlls\_method = 3}, specifies the shift to be used in the More-Sorensen method.

\itt{double more\_sorensen\_tiny} if {\tt nlls\_method = 3}, specifies the value
below which we consider numbers to be essentially zero.

\itt{double more\_sorensen\_tol} if {\tt nlls\_method = 3}, specifies the tolerance
to be used in the More-Sorensen method.

\itt{double hybrid\_tol} if \(\|J^T f \| < \mathtt{hybrid\_tol} * 0.5 \|f\|^2\), switches to a (quasi-)Newton method.

\itt{int hybrid\_switch\_its} sets how many iterates in a row must
the condition in the definition of {\tt hybrid\_tol} hold before a switch.

\itt{bool output\_progress\_vectors} if true, outputs the progress vectors at the end of the routine.

\end{description}


\subsection{The derived data type for holding information}
\label{typeinform}
The structure of type {\tt struct ral\_nlls\_inform} is used
to hold information from the execution of {\tt ral\_nlls}.
The members are:
\begin{description}
\itt{int status} gives the exit status of the subroutine.  See Section~\ref{hslerrors} for details.
\itt{char error\_message[81]} holds the error message corresponding to the exit status.
\itt{int alloc\_status} gives the status of the last attempted allocation/deallocation.
\itt{char bad\_alloc[81]} holds the name of the array that was being allocated when an error was flagged.
\itt{int iter} gives the total number of iterations performed.
\itt{int f\_eval} gives the total number of evaluations of the objective function.
\itt{int g\_eval} gives the total number of evaluations of the gradient of the objective function.
\itt{int h\_eval} gives the total number of evaluations of the Hessian of the objective function.
\itt{int convergence\_normf} tells us if the test on the size of \(f\) is satisfied.
\itt{int convergence\_normg} tells us if the test on the size of the gradient is satisfied.
%\itt{double *resvec} holds the vector of residuals
%\itt{double *gradvec} holds the vector of gradients.
\itt{double obj} holds the value of the objective function at the best estimate of the solution determined by the algorithm.
\itt{double norm\_g} holds the 2-norm of the  gradient of the objective function at the best estimate of the solution determined by the algorithm.
\itt{double scaled\_g} holds the norm gradient of the objective function at the best estimate of the solution determined by the algorithm, where the norm used is the matrix norm associated with the scaling matrix.
\itt{int external\_return} gives the error code that was returned by a call to an external routine.
\itt{char external\_name[81]} holds the name of the external code that flagged an error.
\end{description}

%%%%%%%%%%%%%%%%%%%%%% Warning and error messages %%%%%%%%%%%%%%%%%%%%%%%%

\hslerrors

A successful return from a subroutine in the package is indicated by
{\tt inform.status} having the value zero.
A non-zero value is associated with an error message that by default will
be output on the Fortran unit {\tt inform.error}.  This string is also passed to the 
calling routine in {\tt inform.error\_message}.

% Error and warning messages -- same for C and Fortran

Possible values are:
\begin{description}
\item{} {\tt -1} Maximum number of iterations reached without convergence.
\item{} {\tt -2} Error from evaluating a function/Jacobian/Hessian.
\item{} {\tt -3} Unsupported choice of model.
\item{} {\tt -4} Error return from an external routine.
\item{} {\tt -5} Unsupported choice of method.
\item{} {\tt -6} Allocation error.
\item{} {\tt -7} Maximum number of reductions of the trust radius reached.
\item{} {\tt -8} No progress being made in the solution.
\item{} {\tt -9} \texttt{n}$>$\texttt{m}.
\item{} {\tt -10} Unsupported trust region update strategy.
\item{} {\tt -11} Unable to valid step when solving trust region subproblem.
\item{} {\tt -12} Unsupported scaling method.
\item{} {\tt -101} Unsupported model in dogleg (\texttt{nlls\_method = 1}).
\item{} {\tt -201}  All eigenvalues are imaginary (\texttt{nlls\_method=2}).
\item{} {\tt -202} Matrix with odd number of columns sent to \texttt{max\_eig} subroutine (\texttt{nlls\_method=2}).
\item{} {\tt -301} {\tt nlls\_options\ct more\_sorensen\_max\_its} is exceeded in \texttt{more\_sorensen} subroutine (\texttt{nlls\_method=3}).
\item{} {\tt -302} Too many shifts taken in \texttt{more\_sorensen} subroutine (\texttt{nlls\_method=3}).
\item{} {\tt -303} No progress being made in \texttt{more\_sorensen} subroutine (\texttt{nlls\_method=3}).
\item{} {\tt -401} {\tt nlls\_options\ct model = 4} selected, but {\tt nlls\_options\ct exact\_second\_derivatives} is set to {\tt false}.
\end{description}


\hslgeneral

\hslrestrictions {\tt m$\ge$n$\ge$1}.

\hslmethod
\label{method}

% describe the method we use

The algorithm uses a trust-region method to minimize the cost function
$F(\vx)$ \ref{eq:nlls_problem}.  From a starting point $\vx_0$, the algorithm finds a direction $\vd$ that minimizes a quadratic model of the $F(\vx)$.

The subroutine \texttt{nlls\_solve} simply calls the subroutine 
\texttt{nlls\_iterate} in a loop until either the number of iterates reaches 
\texttt{control\ct maxit}, or a convergence test is satisfied.

The algorithm used by \texttt{nlls\_iterate} is given as Algorithm~\ref{alg:nlls_iterate}.
This is a trust region method that calculates and returns step $\vd$ that 
reduces the model by an acceptable amount.  

\begin{algorithm}
\caption{nlls\_solve}
\label{alg:nlls_solve}
  \begin{algorithmic}
    \State $ {\tt {\bf function} \  \tx = nlls\_solve}(\iter[0]{\tx},{\tt options})$
    \State $\iter[0]{\tr} = {\tt eval\_r}(\iter[0]{\tx})$, $\iter[0]{\tJ} = {\tt eval\_J}(\iter[0]{\tx})$
    \Comment Evaluate residual and Jacobian at initial guess
    \State $\Delta = {\tt options\ct initial\_radius}$
    \State $ \iter[0]{\tg} = - {\iter[0]{\tJ}}^T\iter[0]{\tr}$
    \If {second-order information needed for model}
      \If { {\tt options\ct exact\_second\_derivatives} }
        \State $\iter[0]{\thess} = {\tt eval\_HF(\iter[0]{\tx},\iter[0]{\tr})}$
      \Else
        \State $\iter[0]{\thess} = {\tt 0}$
      \EndIf
    \EndIf
    \For { $k = 0, \dots, {\tt options\ct maxit}$}
      \While{ ${\tt success} \ne  1$ } 
        \State Calculate a potential step $\td$
        \State $\iter[k+1]{\tx} = \iter[k]{\tx} + \td$
        \State $\iter[k+1]{\tr} = {\tt eval\_r}(\iter[k]{\tx})$
        \Comment Evaluate the residual at the new point
        \State $\rho = 0.5 (\| \iter[k+1]{\tr} \|^2 - \| \iter[k]{\tr}\|^2)/(m_k(0) - m_k(\td)) $ 
        \Comment If model is good, $\rho$ should be close to one
          \If{ $\rho > {\tt control\ct eta\_successful}$}
          \State ${\tt success} = 1$
        \EndIf
        \State $\Delta = {\tt update\_trust\_region\_radius}(\Delta,\rho)$
      \EndWhile
      \State $\iter[k+1]{\tJ} = {\tt eval\_J}(\iter[k+1]{\tx})$
      \Comment Evaluate the Jacobian at the new point
      \State $\iter[k+1]{\tg} = -{\iter[k+1]{\tJ}}^T\iter[k+1]{\tr}$
      \If { ${\tt options\ct exact\_second\_derivatives \ne true}$ } 
      \EndIf
      \If {second-order information needed for model}
      \If { {\tt options\ct exact\_second\_derivatives} }
        \State $\iter[k+1]{\thess} = {\tt eval\_HF(\iter[0]{\tx},\iter[0]{\tr})}$
      \Else
        \State $\ty = \iter[k]{\tg} - \iter[k+1]{\tg}$
        \State $\ty^\sharp = {\iter[k]{\tJ}}^T \iter[k+1]{\tr} - \iter[k+1]{\tg}$
        \State $\widehat{\iter[k]{\thess}} = \min\left( 1, \frac{|\td^T\ty^\sharp|}{|\td^T\iter[k]{\thess}\td|}\right) \iter[k]{\thess}$
        \State $\iter[k+1]{\thess} = \widehat{\iter[k]{\thess}} + 
        \left(({\iter[k+1]{\ty}}^\sharp - \iter[k]{\thess}\td 
          )^T\td\right)/\ty^T\td$
      \EndIf
    \EndIf
    \EndFor
  \end{algorithmic}
  
\end{algorithm}




%%% Local Variables: 
%%% mode: latex
%%% TeX-master: "nlls_fortran"
%%% End: 

\hslexample

% setup the example problem

Consider fitting the function $y(t) = x_1e^{x_2 t}$ to data $(\bm{t}, \bm{y})$
using a non-linear least squares fit.\\
The residual function is given by
$$
   r_i(\vx; t_i, y_i) = x_1 e^{x_2 t_i} - y_i.
$$
We can calculate the Jacobian and Hessian as
$$
   J_i(\vx; t_i, y_i) = \left(\begin{array}{cc}
      e^{x_2 t_i} &
      t_i x_1 e^{x_2 t_i}
      \end{array}\right)
$$
$$
   H_i(\vx; t_i, y_i) = \left(\begin{array}{cc}
      1                 & t_i e^{x_2 t_i}    \\
      t_i e^{x_2 t_i}   & t_i^2 e^{x_2 t_i}
   \end{array}\right)
$$
Given the data
\begin{center}
   \begin{tabular}{l|*{5}{r}}
      $i$   & 1 & 2 & 3  & 4  & 5 \\
      \hline
      $t_i$ & 1 & 2 & 4  & 5  & 8 \\
      $y_i$ & 3 & 4 & 6 & 11 & 20
   \end{tabular}
\end{center}
and initial guess $\vx = (2.5, 0.25)$, the following code performs the fit.

\verbatiminput{../example/C/nlls_example.c}


\end{document}
