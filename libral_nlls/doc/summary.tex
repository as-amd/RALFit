% The beginning summary section
{\tt \fullpackagename} computes a solution to the non-linear least-squares problem
\begin{equation}
\min_\vx \  F(\vx) := \frac{1}{2}\| \vr(\vx) \|_{\vW}^2,
\label{eq:nlls_problem}
\end{equation}
where $\vW\in\mathbb{R}^{m\times m}$ is a diagonal, non-negative, weighting matrix, and $\vr(\vx) =(\comp[1]{r}(\vx), \comp[2]{r}(\vx),...,\comp[m]{r}(\vx))^T$ is the non-linear residual of
some data that needs fitting.
% the fit of the data $y$ to some non-linear function ${\bm f} : \mathbb{R}^n \rightarrow \mathbb{R}^m$
% ($m>n$).
% The $n$ variables that are fitted are $\vx=(x_1,x_2,...,x_n)^T$.
% \textcolor{blue}{Some confusion: the $y_i$ don't appear again.}

The algorithm is iterative.
At each point, $\iter{\vx}$, a quadratic model of the function
\[
F(\iter{\vx}) \approx \iter{m}(\vx) := {\iter{\vg}}^T\vx + \frac{1}{2} {\vx}^T \iter{\vH} \vx.
\]
is built.
The `ideal' values in this model are $\iter{\vg} = {\iter{\vJ}}^T\vW\vr(\iter{\vx})$ and
$\iter{\vH} = {\iter{\vJ}}^T\vW\iter{\vJ} + \sum_{i = 1}^m \comp{r}(\iter{\vx}) \vW\nabla^2 \comp{r}(\iter{\vx})$,
where $\iter{\vJ}$ denotes the $m \times n$ Jacobian of $\vr(\vx)$ at the point $\iter{\vx}$
and $\iter{\vH}$ is the Hessian at $\iter{\vx}$.

Once the model has been formed, a trust-region sub-problem of the form
\[
\vd = \arg \min_{\vx} \ \iter{m} (\vx) \quad \mathrm{s.t.} \quad  \|\vx\|_B \leq \Delta,
\]
is solved,
where $\Delta$ is the trust region radius and $B$ is a symmetric positive definite weighting matrix.
The quantity
\[\rho = \frac{F(\iter{\vx}) - F(\iter{\vx} + \vd)}{\iter{m}(\iter{\vx}) - \iter{m}(\iter{\vx} + \vd)}\]
is then calculated.
If this is sufficiently large, the point is accepted and  $\iter[k+1]{\vx}$ is set to $\iter{\vx} + \vd$; if not, the trust-region radius, $\Delta$,
is reduced and  the resulting new trust-region sub-problem is solved.  If the step is very successful -- in that $\rho$ is close to one --
the trust-region radius is increased.

This process continues until either the residual, $\|\vr(\iter{\vx})\|_\vW$, or a measure of the gradient,
$\|{\iter{\vJ}}^T\vW\vr(\iter{\vx})\|_2 / \|\vr(\iter{\vx})\|_\vW$, is sufficiently small.

%%% Local Variables: 
%%% mode: latex
%%% TeX-master: "nlls_fortran"
%%% End: 
